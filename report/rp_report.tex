\documentclass{scrartcl}
\usepackage{natbib}
\usepackage{amsmath, amsfonts, amssymb, bbm}
\numberwithin{equation}{section}
\bibliographystyle{unsrtnat}

%opening
\title{Comparing Hamiltonian Monte Carlo and Elliptical Slice Sampling for constrained Gaussian distributions}
\subtitle{732A76 Research Project Report}
\author{Bayu Brahmantio (baybr878)}

\begin{document}

\maketitle

\section{Background}
High-dimensional multivariate gaussian distribution is used in various models and applications. In some cases, we need to generate from a certain distribution which applies constraints to a multivariate Gaussian distribution (\cite{gelfand1992GS} and \cite{RodrguezYam2004EfficientGS}). Sampling from this distribution is still a challenging issue, particularly because it is not straightforward to compute the normalizing constant for the density function.  

The gibbs sampler has proven to be a suitable choices to sample from truncated multivariate Gaussian distributions (\cite{gelfand1992GS}). Recently, more sophisticated methods have been developed to generate samples from truncated multivariate Gaussian distributions. In this research project, two methods, namely Exact Hamiltonian Monte Carlo (\cite{pakman2013exact}) and Analytic Elliptical Slice Sampling (\cite{Fagan2016ESSwEP}), will be compared. 


\section{Definitions}
\subsection{Truncated Multivariate Gaussian Distribution}
The truncated multivariate Gaussian distribution is a probability distribution obtained from a multivariate Gaussian random variable by bounding it under some linear (or quadratic) constraints.   

Let $\textbf{w}$ be a $d$-dimensional Gaussian random variable with mean vector $\boldsymbol{\mu}$ and covariance matrix $\boldsymbol{\Sigma}$. The corresponding truncated multivariate Gaussian distribution can be defined as
\begin{equation}\label{eq:tmg}
	p(\boldsymbol{\textbf{x}}) = \frac{\exp\{-\frac{1}{2}(\textbf{x}-\boldsymbol{\mu})^{\intercal} \boldsymbol{\Sigma}^{-1}(\textbf{x}-\boldsymbol{\mu})\}}{\int_{\textbf{F}\textbf{x} + \textbf{g} \geq 0} \exp\{-\frac{1}{2}(\textbf{x}-\boldsymbol{\mu})^{\intercal} \boldsymbol{\Sigma}^{-1}(\textbf{x}-\boldsymbol{\mu})\} d\textbf{x}}\mathbbm{1}(\textbf{F}\textbf{x} + \textbf{g} \geq 0)
\end{equation}
where $\textbf{x}$ is a $d$-dimensional truncated Gaussian random variable, $\mathbbm{1}$ is an indicator function, and $\textbf{F}$ is an $m \times d$ matrix, which, together with the $m \times 1$ vector of $\textbf{g}$, defines all $m$ constraints of $p(\boldsymbol{\textbf{x}})$.  We denote this as $\textbf{x} \sim TN(\boldsymbol{\mu}, \boldsymbol{\Sigma};\textbf{F},\textbf{g})$.   

We can rewrite $p(\boldsymbol{\textbf{x}})$ as
\begin{equation}\label{eq:tmg2}
	p(\textbf{x}) = \frac1Z\exp\bigg\{-\frac{1}{2}\textbf{x}^{\intercal} \boldsymbol{\Lambda} \textbf{x} + \boldsymbol{\nu}^{\intercal}\textbf{x}\bigg\}\mathbbm{1}(\textbf{F}\textbf{x} + \textbf{g} \geq 0)
\end{equation}
where $\boldsymbol{\Lambda} = \boldsymbol{\Sigma}^{-1}$, $\boldsymbol{\nu} = \boldsymbol{\Sigma}^{-1}\boldsymbol{\mu}$, and $Z$ is the normalizing constant. Through linear change of variables, \eqref{eq:tmg2} can be transformed into
\begin{equation}\label{eq:tmg3}
	p(\textbf{x}) = \frac1Z\exp\bigg\{-\frac{1}{2}\textbf{x}^{\intercal}\textbf{x}\bigg\}\mathbbm{1}(\textbf{F}^*\textbf{x} + \textbf{g}^* \geq 0)
\end{equation}
such that $\textbf{x} \sim TN(\textbf{0}, \textbf{I}_d;\textbf{F}^*,\textbf{g}^*)$, for some values of $\textbf{F}^*$ and $\textbf{g}^*$.

\subsection{Exact Hamiltonian Monte Carlo for Truncated Multivariate Gaussians}   
Exact Hamiltonian Monte Carlo (HMC) for Truncated Multivariate Gaussians (TMG) (\cite{pakman2013exact}) considers the exact paths of particle trajectories in a Hamiltonian system
\begin{equation}\label{eq:hml}
	H(\textbf{x}, \textbf{s}) = U(\textbf{x}) + K(\textbf{s})
\end{equation}
where $U(\textbf{x})$ is the potential energy term as a function of particle's position ($\textbf{x}$) and $K(\textbf{s})$ is the kinetic energy term as a function of particle's momentum ($\textbf{s}$). Both $\textbf{x}$ and $\textbf{s}$ are of $d$-dimensions.   
The change of position and momentum over time $t$ can be described by Hamilton's equations
\begin{equation}\label{eq:heqs}
\begin{split}
	\frac{\partial x_i}{\partial t} & = \frac{\partial H}{\partial s_i} \\
	\frac{\partial s_i}{\partial t} & = -\frac{\partial H}{\partial x_i}, \qquad i=1,...,d.\\
\end{split}
\end{equation}

The target distribution is related to the current energy state of the particle through canonical distribution:
\begin{equation}\label{eq:can}
	p(\textbf{x}) \propto \exp\{-E(\textbf{x})\}
\end{equation}
where the target distribution, $p(\textbf{x})$, depends on the value of energy function $E(\textbf{x})$. In a Hamiltonian system, we have $H(\textbf{x}, \textbf{s})$ as our energy function, which results in the canonical distribution:
\begin{equation}\label{eq:can2}
\begin{split}
	p(\textbf{x}, \textbf{s}) &\propto \exp\{-H(\textbf{x}, \textbf{s})\} \\
	&\propto \exp\{-U(\textbf{x})\} \exp\{-K(\textbf{s})\} \\
	&\propto p(\textbf{x})p(\textbf{s}).
\end{split}
\end{equation}

Hence, $\textbf{x}$ and $\textbf{s}$ are independent. To sample from the target distribution $p(\textbf{x})$, we can sample from the joint distribution $p(\textbf{x}, \textbf{s})$ and ignore the variable $\textbf{s}$.  

Suppose our target distribution $p(\textbf{x})$ is a truncated multivariate Gaussian distribution as in ($\ref{eq:tmg3}$). We can set our momenta to be normally distributed, that is $\textbf{s} \sim \mathcal{N}(\textbf{0}, \textbf{I}_d)$. Therefore, the Hamiltonian system can be described as: 
\begin{equation}\label{eq:hsys}
	H = U(\textbf{x}) + K(\textbf{s}) = \frac{1}{2}\textbf{x}^{\intercal}\textbf{x} + \frac{1}{2}\textbf{s}^{\intercal}\textbf{s}
\end{equation}
subject to:  
\begin{equation}\label{eq:st}
	\textbf{F}\textbf{x} + \textbf{g} \geq 0.
\end{equation}
for some values of $\textbf{F}$ and $\textbf{g}$.   

The equations of motion for the Hamiltonian system in \eqref{eq:hsys} are:    
\begin{equation}\label{eq:heqs2}
\begin{split}
	\frac{\partial x_i}{\partial t} & = \frac{\partial H}{\partial s_i} = s_i \\
	\frac{\partial s_i}{\partial t} & = -\frac{\partial H}{\partial x_i} -x_i, \qquad i=1,...,d.\\
\end{split}
\end{equation}

In this sense, we want the particles in Hamiltonian system to only move around inside the constrained space. The exact trajectory of a particle using the equations above is:    
\begin{equation}\label{eq:path}
	x_i(t) = s_i(0)\sin(t) + x_i(0)\cos(t).
\end{equation}

A particle will follow the trajectory above until it hits a wall, or in other words, until $\textbf{F}\textbf{x} + \textbf{g} = 0$. Let $t_h$ be the time when the particle hits wall $h$, or when $\textbf{F}_h \cdot \textbf{x}(t_h) + \text{g}_h = 0$. It will hit the wall with velocity $\dot{\textbf{x}}(t_h)$ which can be decomposed into:   
\begin{equation}\label{eq:bounce}
	\dot{\textbf{x}}(t_h) = proj_{\textbf{n}}\dot{\textbf{x}}(t_h) + proj_{\textbf{F}_h}\dot{\textbf{x}}(t_h)
\end{equation}
where $proj_{\textbf{n}}\dot{\textbf{x}}(t_h)$ is the projection of $\dot{\textbf{x}}(t_h)$ on the normal vector $\textbf{n}$ perpendicular to $\textbf{F}_h$ and
\begin{equation}\label{eq:proj}
\begin{split}
	proj_{\textbf{F}_h}\dot{\textbf{x}}(t_h) &= \frac{\textbf{F}_h \cdot \dot{\textbf{x}}(t_h)}{||\textbf{F}_h||}\frac{\textbf{F}_h}{||\textbf{F}_h||}\\  
	&= \frac{\textbf{F}_h \cdot \dot{\textbf{x}}(t_h)}{||\textbf{F}_h||^2}\textbf{F}_h \\
	&= \alpha_h\textbf{F}_h.
\end{split}
\end{equation}

By inverting the direction of $proj_{\textbf{n}}\dot{\textbf{x}}(t_h)$, we can obtain the reflected velocity as
\begin{equation}\
\begin{split}
	\dot{\textbf{x}}_R(t_h) & = -proj_{\textbf{n}}\dot{\textbf{x}}(t_h) + proj_{\textbf{F}_h}\dot{\textbf{x}}(t_h) \\
	& = -\dot{\textbf{x}}(t_h) + 2\alpha_h\textbf{F}_h
\end{split}
\end{equation}
which can be used as the new initial velocity ($s_i(0)$) in \eqref{eq:path} for the particle to continue its path.


\bibliography{ref}

\end{document}
